\chapter{Introducere}

\section{Context și motivație}
EEG-ul, sau Electroencefalograma, reprezintă o modalitate sigură și non-invazivă de a măsura activatea electromagnetică a creierului. Această măsurare se face printr-o cască ce conține mai mulți electrozi. Fiecare electrod în parte măsoara activiatea creierului in Volți. Ele au aplicații intr-o multitudine de domenii. Cele mai folosite cazuri în literatura de specialitate sunt: detectarea fazelor somnului, a simptomelor epilepsiei, a sindromului Alzheimer, a diferitelor emoții puternice, precum tristețe, frică, fericire și, în domeniul de Brain-Computer interface, care se concentrează pe decodarea semnalelor creierului asociate cu mișcarea membrelor.
De-a lungul anilor, cercetătorii au obținut rezultate din ce in ce mai bune în domeniul clasificarii emoțiilor induse de un stimul vizual. Daca în anul 2016 modelele state of the art ce se bazau pe SVM-uri ajungeau la o acuratețe de 73\%\cite{ATKINSON201635}, în anul 2020, modele precum TSception, ce se bazează pe Rețele neuronale ating acuratețe de peste 86\%\cite{TSception}. 

Totuși, în majoritatea cazurilor, stimulii induși sunt cauzați fie de evenimente de lungă durată: video-uri, ore intregi de somn, fie de evenimente binare, având clase target sau non-target.

\section{Obiectivele Lucrării}
Scopul lucrării este de a vedea dacă putem clasifica stimuli vizuali, prezentați pe parcursul unei durate scurte, în funcție de reacția unui participant, dispunând de 3 categorii de răspuns: pozitiv, neutru, negativ.
