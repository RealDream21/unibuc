\chapter{Introducere}

\section{Context și motivație}
EEG-ul, sau Electroencefalograma, reprezintă o modalitate sigură și non-invazivă de a măsura activatea electromagnetică a creierului. Această măsurare este realizată printr-o cască ce conține mai mulți electrozi. Fiecare electrod în parte măsoara activiatea creierului in Volți. Aceste metode au aplicații intr-o multitudine de domenii. Cele mai folosite cazuri în literatura de specialitate sunt: detectarea fazelor somnului, a simptomelor epilepsiei, a sindromului Alzheimer, a diferitelor emoții puternice, precum tristețe, frică, fericire și, în domeniul de Brain-Computer interface, care se concentrează pe decodarea semnalelor creierului asociate cu mișcarea membrelor.
De-a lungul anilor, cercetătorii au obținut rezultate din ce în ce mai bune în domeniul clasificării emoțiilor induse de un stimul vizual. Dacă în anul 2016 cele mai bune modele se bazau pe SVM-uri și ajungeau la o acuratețe de 73\% \cite{ATKINSON201635} pentru clasificări între două clase, și 62\% pentru clasificări între trei clase, în anul 2020, modele precum TSception, ce se bazează pe rețele neuronale ating acurateți de peste 86\% \cite{TSception}. 

Totuși, în majoritatea cazurilor, stimulii induși sunt cauzați fie de evenimente de lungă durată: video-uri, ore întregi de somn, fie de evenimente binare, având clase target/non-target, de exemplu în cazul visual speller-elor \cite{visual_speller}. Mi-am propus să stabilesc dacă stimulii de scurtă durată au același potențial de a evoca răspunsuri capabile de a fi interpretate algoritmic precum cei de lungă durată în contextul unui grup de arhitecți care privesc coloane. Privirea unei coloane nu provoacă reacții emoționale puternice, dar tot are potențialul de a stârni anumite tipare de răspuns la nivelul creierului.

\section{Obiectivele Lucrării}
Scopul lucrării este de a prezice reacția subconștientă pe care o are mintea unui arhitect, provocată de afișarea pe durată scurtă a unor imagini reprezentând coloane generate artificial. În acest context, datele reprezintă semnale EEG de tip ERP (Event-Related Potential). În plus, prezicerea este făcută de-a lungul a trei dimensiuni afective: valență, excitare și dominanță. Abordarea are la bază presupunerea că ERP-urile reflectă reacții spontane, și faptul că, dacă un participant se gândește mai mult timp la o imagine începe să nu mai fie sigur dacă îi place sau nu. Spre îndeplinirea acestui scop, am folosit cât mai mulți algoritmi, comparându-i între ei pentru fiecare paradigmă de interpretare a semnalelor.

\section{Noțiuni preliminare despre paradigma ERP}

ERP (Event-Related Potential), sau, în traducere, potențial legat de eveniment, reprezintă răspunsul creierului uman asociat unui eveniment stimulant \cite{erp_introduction}. Evenimentele pot fi cauzate de diverși astfel de stimuli, de exemplu auditivi sau vizuali. Aceștia sunt captați prin intermediul unei căști EEG ce înregistrează activitatea creierului. ERP-urile pot fi mai departe împărțite în mai multe categorii în funcție de semnificația pe care dorim să o extragem din semnal. ERP-urile reprezintă în sine o fereastră de timp din semnalul original unde se află răspunsul subconștient al subiectului în fața stimulului prezentat. Înăuntrul acestor ferestre de timp pot apărea componente negative sau pozitive care variază semnificativ față de medie. Aceste componente sunt fie notate cu P (pentru vârfuri pozitive), sau cu N (pentru vârfuri negative). Datele pe care se bazează experimentele mele sunt legate de evenimente de tip P300, care, din denumire, reprezintă un vârf pozitiv ce apare după 300 milisecunde de la prezentarea stimulului.