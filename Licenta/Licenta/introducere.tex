\chapter{Introducere}

\section{Context și motivație}
EEG-ul, sau Electroencefalograma, reprezintă o modalitate sigură și non-invazivă de a măsura activatea electromagnetică a creierului. Această măsurare este realizată printr-o cască ce conține mai mulți electrozi. Fiecare electrod în parte măsoara activiatea creierului in Volți. Aceste metode au aplicații intr-o multitudine de domenii. Cele mai folosite cazuri în literatura de specialitate sunt: detectarea fazelor somnului, a simptomelor epilepsiei, a sindromului Alzheimer, a diferitelor emoții puternice, precum tristețe, frică, fericire și, în domeniul de Brain-Computer interface, care se concentrează pe decodarea semnalelor creierului asociate cu mișcarea membrelor.
De-a lungul anilor, cercetătorii au obținut rezultate din ce in ce mai bune în domeniul clasificării emoțiilor induse de un stimul vizual. Dacă în anul 2016 modelele state of the art ce se bazau pe SVM-uri ajungeau la o acuratețe de 73\%\cite{ATKINSON201635}, pentru clasificări între două clase și 62\% pentru clasificări între trei clase, în anul 2020, modele precum TSception, ce se bazează pe rețele neuronale ating acuratețe de peste 86\%\cite{TSception}. 

Totuși, în majoritatea cazurilor, stimulii induși sunt cauzați fie de evenimente de lungă durată: video-uri, ore intregi de somn, fie de evenimente binare, având clase target/non-target în cazul, de exemplu, al visual speller-elor\cite{visual_speller}. Ne-am propus să vedem dacă stimulii de scurtă durată au același potențial de a evoca răspunsuri capabile de a fi interpretate algoritmic.

\section{Obiectivele Lucrării}
Scopul lucrării este de a vedea dacă putem clasifica stimuli vizuali, prezentați pe parcursul unei durate scurte, în funcție de reacția unui participant, dispunând de 3 categorii de răspuns: pozitiv, neutru, negativ. În încercarea de a îndeplini acest scop, am folosit cât mai mulți algoritmi disponibili, comparându-i între ei pentru fiecare paradigma.
