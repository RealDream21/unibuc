\begin{abstractpage}

\begin{abstract}{romanian}
Creierul uman constituie una dintre cele mai complexe și mai promițătoare arii de cercetare. Și, contrar faptului că a fost studiat intens de-a lungul timpului, încă nu avem un model destul de clar al modului în care acesta funcționează.

În această lucrare de licență am urmărit să explorez, să evaluez și să compar cât mai mulți algoritmi legați de interpretarea semnalelor emise de creier. Scopul principal este de a vedea gradul de interpretabilitate generat de diverși stimuli de scurtă durată. Lucrarea mea se diferențiază de restul lucrărilor din domeniul ERP prin tipul datelor folosite și prin specificitatea subiecților. Astfel, datele mele provin de la oameni specializați, și anume arhitecți ce au privit imagini specifice domeniului, mai exact coloane generate artificial.

Pe parcursul lucrării am implementat și comparat sistematic algoritmi documentați în literatura de specialitate.
\end{abstract}

\begin{abstract}{english}
The human brain represents one of the most complex and promising areas of research. And, although it has been intensely  studied throughout the years, we still do not have an accurate model of the way it functions.

In this thesis, I aimed to explore, evaluate and compare a wide range of algorithms related to the interpretation of brain-emitted signals. The main objective is to asses the degree of interpretability elicited by various short-duration stimuli. My work differs from the existing literature in the ERP domain via the type of data used and the specificity of the subjects. The data used comes from specialized subjects, more particularly architects that have been shown images specific to their domain, specifically, artificially generated columns.

Throughout the study, I systematically implemented and compared algorithms documented in the scientific literature.
\end{abstract}

\end{abstractpage}