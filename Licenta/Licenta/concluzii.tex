\chapter{Concluzii}

\section{Interpretarea rezultatelor}

Observând rezultatele obținute, putem ajunge la concluzia că stimulii nu au fost destul de puternici astfel încât să putem face o distincție clară între reacția subiecților la aceștia. Iar, dacă privim literatura de specialitate în domeniul EEG-urilor, și în special pe cel al clasificării emoțiilor, putem observa rezultate cel mult medicore\cite{recreation_attempt}, chiar și pe seturi de date bine cunoscute. O analiză mai amplă și precisă poate fi facută, momentan, folosind tehnologie non-invazivă, doar pe stimuli de durată mai lungă, mai puternici și mai diverși. 

\section{Îmbunătățiri și perspective de viitor}

Pentru a îmbunătăți rezultalte, experimentul ar putea fi realuat folosind imagini mai stimulante, afișate pe o perioadă mai lungă de timp. De asemena, suntem de părere că o îmbunătățire ar putea fi adusă și prin folosirea unui eyetracker, astfel combinând reacția, cu zona în care s-a uitat participantul când a avut-o. În plus, un alt aspect care ar putea îmbunătăți rezultalte, ar fi creșterea gradului de imersiune folosind, de exemplu, căști de realitate virtuală.