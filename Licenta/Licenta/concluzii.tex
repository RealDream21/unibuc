\chapter{Concluzii}

\section{Interpretarea rezultatelor}

Observând rezultatele obținute, putem ajunge la concluzia că stimulii nu au fost destul de puternici astfel încât să putem face o distincție clară între reacția subiecților la aceștia. Iar, dacă privim literatura de specialitate în domeniul EEG-urilor, și în special pe cel al clasificării emoțiilor, putem observa rezultate cel mult medicore\cite{recreation_attempt}, chiar și pe seturi de date bine cunoscute. Încercând cât mai multe lucruri și citind lucrări de specialitate am observat că există destule limitări în acest domeniu. Multe dintre limitări sunt legate de metodele de recoltare a datelor și de zgomotul rezultat de acestea, altele sunt legate de diferențele fiziologice dintre oameni, neputând astfel generaliza un stimul unifrom de-a lungul unor subiecți. O analiză mai amplă și precisă poate fi facută, momentan, folosind tehnologie non-invazivă, doar pe stimuli de durată mai lungă, unde cel mai important factor este frecvența, sau stimuli mai puternici și mai diverși. 

\section{Îmbunătățiri și perspective de viitor}

Pentru a îmbunătăți rezultalte, experimentul ar putea fi realuat folosind imagini mai stimulante, afișate pe o perioadă mai lungă de timp. De asemena, încă o îmbunătățire ar putea fi adusă și prin folosirea unui eyetracker, astfel combinând reacția, cu zona în care s-a uitat participantul când a avut-o. În plus, un alt aspect care ar putea îmbunătăți rezultalte, ar fi creșterea gradului de imersiune folosind, de exemplu, căști de realitate virtuală.